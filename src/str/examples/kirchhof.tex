Consider a simple series circuit with a DC power source and two resistors, as shown below, \textbf{assuming the passive convention}:

\begin{figure}[ht]
  \centering
  \begin{circuitikz}[american voltages]
    \draw (0,0) to[V, l=$V_g$, i=$i_g$] (0,2)
          to[R, l=$R_1$, v=$V_1$] (2,2)
          to[R, l=$R_2$, v=$V_2$] (4,2)
          -- (4,0) -- (0,0);
  \end{circuitikz}
  \caption{Series circuit with a DC power source and two resistors}
\end{figure}

According to Kirchhoff's Voltage Law (KVL), the sum of the voltages around any closed loop in the circuit is equal to zero. For this circuit, the equation is:

\begin{equation}
V_g = V_1 + V_2
\end{equation}

According to Ohm's law, the voltage across a resistor is equal to the current through the resistor times the resistance of the resistor. So we can write:

\begin{align}
V_1 &= i_g \cdot R_1 \\
V_2 &= i_g \cdot R_2
\end{align}

Substituting these equations into the KVL equation, we get:

\begin{equation}
V_g = i_g \cdot R_1 + i_g \cdot R_2 = i_g \cdot (R_1 + R_2)
\end{equation}

This equation tells us that the voltage of the power source is equal to the current through the circuit times the total resistance of the circuit.

According to Kirchhoff's Current Law (KCL), the sum of the currents entering a node (or a junction) equals the sum of the currents leaving the node. For this circuit, the equation is:

\begin{equation}
i_g = i_1 = i_2
\end{equation}

This equation tells us that the current through the power source is equal to the current through each resistor, which is a characteristic of series circuits.

It's important to note that the direction of the loop does not affect the application of KVL. Whether we move in the direction of the current (from $+$ to $-$) or against it (from $-$ to $+$), the sum of the voltages around the loop is still zero. This is true even in the presence of a power source or a load, where the power is negative or positive, respectively.
