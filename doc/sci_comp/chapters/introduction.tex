Para fins de simplicidade, dividi o ciclo de vida de um o desenvolvimento de um 
software em cinco grandes etapas: {\textbf{\textit{A - Concepção, Pré-Requisitos e Documentação};
\textit{B - Design do Software}; \textit{C - Desenvolvimento e Implementaçãp do Software}; 
\textit{D - Testes}; e \textit{E - Implementação e Manutenção}.}} \\

Cada uma dessas etapas é composta por várioas sub-etapas, que serão detalhadas no decorrer
deste documento. Por se tratar de um documento informal, fruto de nota de estudos minhas,
me abstive de formalizar as referencias bibliográficas consultadas como em livros e artigos
científico; mas, sempre que oportuno, referências serão citadas diretamente no corpo do texto,
e todo o meu material de uso pessoal está disponível no meu repositório em GitHub. \\

Minha motivação para escrever este documento é a de organizar o que venho estudando ao longo dos anos 
sobre programação e computação científica - desde o momento em que os problemas eram hoje pra mim triviais,
como "qual a melhor linguagem de programação para resolver problemas de engenharia"; à questões mais complexas,
como estruturas de dados, arquitetura de software, análise de pré-requisitos, escalabilidade, manutenção, segurança,
e, principalmente, se a aplicação que penso em desenvolver deve, de fato, ser desenvolvida. \\

Também, espero que este documento, de livre acesso e código aberto, posssa motivar mais profissionais que não são formalmente
de alguma área da computação, como Bacharéis em Ciências da Computação ou Bacharéis em Engenharia de Computação, a se aprofundarem,
mas que tem profunda necessidade de uso e execução de recursos computacionais em nível de software e hardware 
para que seu trabalho possa acontecer - literalmente, \textbf{poder acontecer!} \\

Na medida do possível, por se tratar de um documento que registra estudo e anotações de computação voltada
à STEM - Science, Technology, Engineering and Mathematics - Sciences, tentarei trazer conceitos da Matemática Superior,
Física Teórica e Aplicada, Engenharias, Epidemiologia, Estatística e outras áreas de nível superior que, apesar de contemplarem
a computação em seus currículos, não são formalmente de computação. \\

\textbf{LH}.
